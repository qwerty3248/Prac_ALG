\documentclass{article}
\usepackage{amsmath}
\usepackage{listings}
\usepackage{color}
\usepackage [latin1]{inputenc}

\definecolor{mygreen}{rgb}{0,0.6,0}
\definecolor{mygray}{rgb}{0.5,0.5,0.5}
\definecolor{mymauve}{rgb}{0.58,0,0.82}

\lstset{
  backgroundcolor=\color{white},
  basicstyle=\ttfamily\footnotesize,
  breakatwhitespace=false,
  breaklines=true,
  captionpos=b,
  commentstyle=\color{mygreen},
  deletekeywords={...},
  escapeinside={\%*}{*)},
  extendedchars=true,
  frame=single,
  keepspaces=true,
  keywordstyle=\color{blue},
  language=C++,
  otherkeywords={*,...},
  numbers=left,
  numbersep=5pt,
  numberstyle=\tiny\color{mygray},
  rulecolor=\color{black},
  showspaces=false,
  showstringspaces=false,
  showtabs=false,
  stepnumber=1,
  stringstyle=\color{mymauve},
  tabsize=2,
  title=\lstname
}

\begin{document}

\title{Resoluci\'on de Planificaci\'on de Viajes A\'ereos con Programaci\'on Din\'amica}
\author{Jes\'us Losada Arauzo \and Javier G\'omez Mole\'on}
\date{\today}
\maketitle

\section*{1. Dise\~no de resoluci\'on por etapas y ecuaci\'on recurrente}

El problema puede ser resuelto en etapas donde cada etapa considera la inclusi\'on de un nodo intermedio $k$. La ecuaci\'on recurrente para la programaci\'on din\'amica ser\'ia:

\[
D[i][j] = \min(D[i][j], D[i][k] + D[k][j] + E(k))
\]

donde $D[i][j]$ es el costo m\'inimo para viajar de $i$ a $j$ pasando por el nodo intermedio $k$.

\section*{2. Dise\~no de la memoria}

Usamos una matriz de tama\~no $n \times n$ para almacenar los costos m\'inimos de viaje entre todas las ciudades $i$ y $j$. La matriz inicializa los valores seg\'un la matriz de tiempo $T$ y ajusta los valores con las actualizaciones de los nodos intermedios.

\section*{3. Verificaci\'on del P.O.B. (Principio de Optimalidad de Bellman)}

El principio de optimalidad de Bellman se verifica mediante la ecuaci\'on recurrente utilizada en el algoritmo de Floyd-Warshall. Esto asegura que el camino m\'as corto entre dos nodos $i$ y $j$ pasando por $k$ es \'optimo.

\section*{4. Dise\~no del algoritmo de c\'alculo de coste \'optimo}

El algoritmo de c\'alculo de coste \'optimo es el algoritmo de Floyd-Warshall modificado para incluir el tiempo de escala $E(k)$.

\section*{5. Dise\~no del algoritmo de recuperaci\'on de la soluci\'on}

Para recuperar la soluci\'on (el camino m\'as corto), necesitamos mantener una matriz de predecesores que nos permita reconstruir el camino.

\section*{6. Implementaci\'on de los algoritmos de c\'alculo de coste \'optimo y recuperaci\'on de la soluci\'on}

\begin{lstlisting}
// Incluyendo bibliotecas necesarias
#include <iostream>
#include <vector>
#include <climits>
#include <algorithm>

using namespace std;

const int INF = INT_MAX;

// Implementaci\'on del algoritmo de Floyd-Warshall adaptado con recuperaci\'on de camino
void floydWarshall(const vector<vector<int>>& T, const vector<int>& E, vector<vector<int>>& D, vector<vector<int>>& P) {
    int n = T.size();
    
    // Inicializaci\'on de la matriz de distancias y predecesores
    for (int i = 0; i < n; ++i) {
        for (int j = 0; j < n; ++j) {
            if (i == j) {
                D[i][j] = 0;
                P[i][j] = -1; // No hay predecesor
            } else if (T[i][j] != 0) {
                D[i][j] = T[i][j];
                P[i][j] = i; // El predecesor de j es i
            } else {
                D[i][j] = INF;
                P[i][j] = -1; // No hay predecesor
            }
        }
    }

    // Actualizaci\'on de la matriz de distancias y predecesores
    for (int k = 0; k < n; ++k) {
        for (int i = 0; i < n; ++i) {
            for (int j = 0; j < n; ++j) {
                if (D[i][k] != INF && D[k][j] != INF && D[i][k] + D[k][j] + E[k] < D[i][j]) {
                    D[i][j] = D[i][k] + D[k][j] + E[k];
                    P[i][j] = P[k][j]; // Actualiza el predecesor
                }
            }
        }
    }
}

// Funci\'on para recuperar el camino m\'as corto
vector<int> getPath(int i, int j, const vector<vector<int>>& P) {
    vector<int> path;
    if (P[i][j] == -1) {
        return path; // No hay camino
    }
    path.push_back(j);
    while (i != j) {
        j = P[i][j];
        path.push_back(j);
    }
    reverse(path.begin(), path.end());
    return path;
}

int main() {
    // Matriz de tiempos T y vector de tiempos de escala E
    vector<vector<int>> T = {
        {0, 1, 3, 4},
        {1, 0, 2, 3},
        {3, 2, 0, 4},
        {4, 3, 4, 0}
    };
    vector<int> E = {1, 1, 1, 1};

    int n = T.size();
    vector<vector<int>(n, vector<int>(n, INF)));
    vector<vector<int>> P(n, vector<int>(n, -1));

    floydWarshall(T, E, D, P);

    // Imprimir la matriz de distancias mínimas
    cout << "Matriz de costos mínimos:" << endl;
    for (const auto& fila : D) {
        for (const auto& valor : fila) {
            if (valor == INF) {
                cout << "INF ";
            } else {
                cout << valor << " ";
            }
        }
        cout << endl;
    }

    // Recuperar y imprimir el camino más corto de ejemplo
    int start = 0, end = 3;
    vector<int> path = getPath(start, end, P);
    cout << "Camino más corto de " << start + 1 << " a " << end + 1 << ": ";
    for (int city : path) {
        cout << city + 1 << " ";
    }
    cout << endl;

    return 0;
}
\end{lstlisting}

\end{document}


